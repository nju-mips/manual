%!TEX program = xelatex
% 完整编译: xelatex -> bibtex -> xelatex -> xelatex
\documentclass[lang=cn,11pt,a4paper]{elegantpaper}

\usepackage{calc}
\usepackage{times}
\usepackage{float}

\usepackage{pdfpages}

\usepackage{amssymb} % \Box
%\usepackage{algorithm}  
%\usepackage{algorithmicx}  
\usepackage{algpseudocode}
\usepackage{amsmath}
\usepackage[linesnumbered,ruled,lined]{algorithm2e}
\usepackage{graphicx}
\usepackage{listings}
\usepackage{color}
\usepackage{colortbl}
\usepackage{longtable}
\SetArgSty{textnormal} % non-italics in if-clause of algorithm

\usepackage{hyperref}
\usepackage{titlesec}
\titleformat{\section}{\centering\Huge\bfseries}{\thesection}{1em}{}
\titleformat{\subsection}{\LARGE\bfseries}{\thesubsection}{1em}{}

\title{计算机系统综合实验手册}
\author{欧先飞 \\ ouxianfei@smail.nju.edu.cn}
%\institute{\href{https://elegantlatex.org/}{Elegant\LaTeX{} 项目组}}

%\newcommand{\blankbox}{\boxed{\textcolor[rgb]{1,1,1}{*}}}
\newcommand{\blankbox}{\Large{\qedsymbol}}

%\version{0.08}
%\date{\zhtoday}

\begin{document}

\maketitle

\tableofcontents

\newpage

\section{基本工具链}

\subsection{Makefile}

\subsubsection{规则依赖}

一个简单的例子:
\begin{lstlisting}
a.o: a.c
  g++ a.c -o a.o

b.o: b.c
  g++ b.c -o b.o

binary: a.o b.o
  echo + binary
\end{lstlisting}

将上述内容写入到你当前目录的Makefile文件中,并保证\lstinline!a.c!和\lstinline!b.c!文件存在,然后在命令行里敲\lstinline!make binary!,这条语句的意思是制造\lstinline!binary!这个目标,make会读特定名称的文件来获取制造这个目标的规则(Makefile是make内置的一个特定名称)。make读取你的Makefile后会发现binary依赖于\lstinline!a.o!和\lstinline!b.o!,并进而计算得出\lstinline!a.o!依赖于\lstinline!a.c!,\lstinline!b.o!依赖于\lstinline!b.c!,最后按照你给定的生成规则\lstinline!g++ a.c -o a.o!和\lstinline!g++ b.c -o b.o!去生成相应的\lstinline!a.o!和\lstinline!b.o!。

敲完\lstinline!make binary!之后,你binary并没有生成,因为你还没有为binary添加生成规则,所以如果你再敲一遍\lstinline!make binary!,相应的\lstinline!echo + binary!语句会被再执行一遍。

我们稍微改一下,让他变成下面这样:
\begin{lstlisting}
a.o: a.c
  g++ a.c -o a.o

b.o: b.c
  g++ b.c -o b.o

binary: a.o b.o
  cat a.o b.o > binary
\end{lstlisting}

这个时候如果你在敲\lstinline!make binary!,生成规则会将\lstinline!a.o!与\lstinline!b.o!拼接成binary文件,由于binary文件是最新生成的,它的时间戳大于\lstinline!a.o!和\lstinline!b.o!的时间戳,所以再下一次你敲\lstinline!make binary!的时候,make会输出\lstinline!'binary' is up to date!。

一般而言,我们并不希望我们所生成的中间文件和源码文件共用一个目录,假设我们希望将目录结构调整成这样:
\begin{lstlisting}
+---src
|   +--- a.c
|   +--- b.c
|
+---build
    +--- a.o
    +--- b.o
    +--- binary
\end{lstlisting}

为了达成这个目的,我们需要对Makefile做一些微调:
\begin{lstlisting}
build/a.o: src/a.c
  mkdir -p build
  g++ src/a.c -o build/a.o

build/b.o: b.c
  mkdir -p build
  g++ src/b.c -o build/b.o

build/binary: build/a.o build/b.o
  cat build/a.o build/b.o > build/binary
\end{lstlisting}

相应的,基于这个Makefile来生成binary,我们需要执行\lstinline!make build/binary!,这条命令会按照你的规则去生成\lstinline!build/a.o!和\lstinline!build/b.o!两个文件,然后拼接成\lstinline!build/binary!这个文件。

这个时候,你可能发现,make的时候连目录一起写进去太撒刁了,所以这个Makefile可以进一步改进:

\begin{lstlisting}
.PHONY: binary

binary: build/binary

build/a.o: src/a.c
  mkdir -p build
  g++ src/a.c -o build/a.o

build/b.o: b.c
  mkdir -p build
  g++ src/b.c -o build/b.o

build/binary: build/a.o build/b.o
  cat build/a.o build/b.o > build/binary
\end{lstlisting}

\lstinline!.PHONY!的作用是定义一个伪目标,什么是伪目标,直观的讲就是,这个目标在磁盘上不存在。之前Makefile里面的\lstinline!a.o!、\lstinline!build/binary!、\lstinline!build/a.o!这些目标在磁盘上是会对应到一个实际存在的文件的,而伪目标不需要满足这个要求,同时相应的,make也不会去检查伪目标的时间戳,而是直接调用他的生成规则,比如:

\begin{lstlisting}

.PHONY: hello

hello:
  touch hello
  echo hello world
\end{lstlisting}

在这个makefile中,无论你敲几次\lstinline!make hello!,\lstinline!touch hello!和\lstinline!echo hello world!这两条生成规则都会被执行。

\subsubsection{通用匹配\%和生成规则简化}

在上文提到的Makefile中,你会发现在编写a.o和b.o的生成规则的时候,你需要把\lstinline!build/!和\lstinline!a.o!这样的多余的东西再写一遍,这个信息明明已经包含在依赖目标和生成目标里了,为了简化规则的编写,makefile定义了一系列的简写符号,用这些符号我们可以将Makefile简化成:

\begin{lstlisting}
.PHONY: binary

binary: build/binary

build/a.o: src/a.c
  mkdir -p $(@D)
  g++ $^ -o $@

build/b.o: b.c
  mkdir -p $(@D)
  g++ $^ -o $@

build/binary: build/a.o build/b.o
  cat $^ > $@
\end{lstlisting}

其中\lstinline!$^!出现在生成规则中表示所有被依赖的目标(示例中\lstinline!$^!会被替换成\lstinline!build/a.o build/b.o!),\lstinline!$@!表示生成的目标(示例中上下两条生成规则里的\lstinline!$@!分别会被替换成\lstinline!build/a.o!和\lstinline!build/b.o!),\lstinline!$(@D)!则是表示生成目标所在的目录,也就是\lstinline!build/!,所以上面的Makefile其实与之前的Makefile完全等价。

\begin{table}[htbp]
\centering
\begin{tabular}{|c|c|}
\hline
符号&含义 \\\hline
\lstinline|$(@D)| & 当前规则目标所在目录 \\\hline
\lstinline|$@| & 当前规则目标 \\\hline
\lstinline|$<| & 当前规则依赖项的第一个 \\\hline
\lstinline|$^| & 当前规则的所有依赖项 \\\hline
\end{tabular}
\end{table}

到这里,你会发现一件更撒刁的事,那就是\lstinline!mkdir -p $(@D)!和\lstinline!g++ $^ -o $@!这两条语句你在两个目标的生成规则中写了两遍,为了解决这样的问题,Makefile又提供了通配符机制,利用通配符,我们可以将Makefile进一步改写成:

\begin{lstlisting}
.PHONY: binary

binary: build/binary

build/%.o: src/%.c
  mkdir -p $(@D)
  g++ $^ -o $@

build/binary: build/a.o build/b.o
  cat $^ > $@
\end{lstlisting}

在这个Makefile上,如果我们敲\lstinline!make binary!,make会去查找\lstinline!build/binary!所依赖的\lstinline!build/a.o!和\lstinline!build/b.o!的生成规则,当然make是找不到的,因为我们并没有明确的写这样的规则。但是make会进一步发现,它只要把\lstinline!build/%.o!中的\lstinline!%!这个字符替换成\lstinline!a!,他就能匹配上\lstinline!build/%.o: src/%.c!这条规则,于是它就把这条规则里面所有的\lstinline!%!都替换成了\lstinline!a!,便得到了\lstinline!build/a.o!的生成规则,同理它也能获得\lstinline!build/b.o!的生成规则。

\subsubsection{变量和函数的使用}

在上面的Makefile里面,我们还有着这样一个问题,就是如果我们不知道src下面究竟有多少\lstinline!.c!文件,但是我们依旧希望Makefile能工作应该怎么办。举个简单的场景,你的项目正在如火如荼的发展中,每天都可能添加若干个\lstinline!.c!文件或重命名若干个\lstinline!.c!文件,如果你每次都要手动更新Makefile,未免太过撒刁了,为了解决这个问题,Makefile提供了函数机制。

回顾我们的需求,我们其实需要的是make能自动找到src下所有的.c文件,并将其重命名为build下的.o文件,最后放到\lstinline!build/binary!的依赖目标中去:

\begin{lstlisting}
.PHONY: binary

binary: build/binary

build/%.o: src/%.c
  mkdir -p $(@D)
  g++ $^ -o $@

build/binary: $(patsubst src/%.c,build/%.o,$(shell find src/ -name "*.c"))
  cat $^ > $@
\end{lstlisting}

新的Makefile中\lstinline!build/binary!所依赖的目标已经被替换成了\lstinline!$(patsubst src/%.c,build/%.o,$(shell find src/ -name "\*.c"))!,我们分开来讲这条语句,首先是\lstinline!$(shell find src/ -name "\*.c")!,这条语句会调用你终端中的\lstinline!find!命令,查找\lstinline!src!目录下所有的\lstinline!.c!文件,并将其替换成查找的结果,也就是说这条语句执行完毕之后原语句会变成\lstinline!$(patsubst src/%.c,build/%.o,src/a.c src/b.c)!,\lstinline!patsubst!是make的一个内置函数,它的作用是根据你指定的初始特征和结果特征,对你的输入进入转化,对应到这条语句,你给它的初始特征是\lstinline!src/%.c!,结果特征是\lstinline!build/%.o!,输入是\lstinline!src/a.c src/b.c!,对于输入中的每一项,如\lstinline!src/a.c!,\lstinline!patsubst!函数会发现将\lstinline!%!换成\lstinline!a!就能匹配上,然后他讲结果特征中的\lstinline!%!也换成\lstinline!a!,并将结果替换掉\lstinline!src/a.c!,也就是说,\lstinline!src/a.c!会被替换成\lstinline!build/a.o!,同理\lstinline!src/b.c!也会被替换成\lstinline!build/b.o!。

上述Makefile看着非常蛋疼,因为我们写了一个超长的语句,有没有一种机制,像编程语言一样,通过定义一些中间变量来拆分超长表达式呢?答案当然是有的,所以上面的Makefile又可以改成下面这样:

\begin{lstlisting}
.PHONY: binary

SRC_DIR := src/
SRCS    := $(shell find $(SRC_DIR) -name "*.o")
OBJ_DIR := build/
OBJS    := $(patsubst src/%.c,build/%.o,$(SRCS))

binary: $(OBJ_DIR)binary

$(OBJ_DIR)%.o: $(SRC_DIR)%.c
  mkdir -p $(@D)
  g++ $^ -o $@

$(OBJ_DIR)binary: $(OBJS)
  cat $^ > $@
\end{lstlisting}

上面的赋值使用了\lstinline!:=!,你暂时不需要知道为什么是\lstinline!:=!而不是\lstinline!=!,如果你真的想了解,在docs目录下的Makefile.pdf可以给你答案,同时如果哪天你需要什么其它的功能,你也可以通过查阅\lstinline!Makefile.pdf!来看看是否有相应的函数。

\subsubsection{宏的使用与自动规则生成}

假如你想把你的Makefile包装成一个功能库,其他人只要include你的Makefile就能获得上述所有的功能,那么你需要怎么做呢?一个简单的答案就是宏。


\section{MIPS相关}

\section{相关项目说明}

\section{运行及移植linux}

\newpage

\nocite{*}
%\bibliographystyle{unsrt}
\bibliography{reference}

\end{document}
